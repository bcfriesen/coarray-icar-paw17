\documentclass[sigconf, authordraft]{acmart}

\usepackage{booktabs} % For formal tables
\usepackage{animate}

\usepackage[acronym,nonumberlist]{glossaries}
\makeglossaries


% Copyright
%\setcopyright{none}
%\setcopyright{acmcopyright}
%\setcopyright{acmlicensed}
\setcopyright{rightsretained}
%\setcopyright{usgov}
%\setcopyright{usgovmixed}
%\setcopyright{cagov}
%\setcopyright{cagovmixed}


% DOI
\acmDOI{xx.xxx/xxx_x}

% ISBN
\acmISBN{xxx-xxxx-xx-xxx/xx/xx}

%Conference
\acmConference[PAW17]{PGAS Applications Workshop}{November 2017}{Denver, Colorado USA}
\acmYear{2017}
\copyrightyear{2017}

%\acmPrice{15.00}

\acmSubmissionID{xxx-xxx-xx}

\input glossary/glossary.tex
\input glossary/acronyms.tex

\begin{document}
\title{Performance portability of an intermediate-complexity atmospheric research model in coarray Fortran}
%\titlenote{Produces the permission block, and
%  copyright information}
%\subtitle{Extended Abstract}
%\subtitlenote{The full version of the author's guide is available as
%  \texttt{acmart.pdf} document}

\author{Damian Rouson}
\orcid{0000-0002-2234-868X}
\affiliation{%
  \institution{Sourcery Institute}
  \streetaddress{2323 Broadway}
  \city{Oakland}
  \state{California}
  \postcode{94612}
}
\email{damian@sourceryinstitute.org}
\renewcommand{\shortauthors}{D. Rouson et al.}

\author{Ethan Gutmann}
\orcid{0000-0003-4077-3430}
\affiliation{%
  \institution{National Center for Atmospheric Research}
  \streetaddress{3450 Mitchell Ln.}
  \city{Boulder}
  \state{Colorado}
  \postcode{80301}
}
\email{gutmann@ucar.edu}

\author{Brian Freisen}
\orcid{0000-0003-4077-3430}
\affiliation{%
  \institution{National Energy Research Scientific Computing Center}
  \streetaddress{1 Cyclotron Road, Mail Stop 59R4010A}
  \city{Berkeley}
  \state{California}
  \postcode{94720}
}
\email{BFriesen@lbl.gov}

\author{Alessandro Fanfarillo}
\orcid{0000-0003-3487-7452}
\affiliation{%
  \institution{National Center for Atmospheric Research}
  \streetaddress{ 1850 Table Mesa Dr.}
  \city{Boulder}
  \state{Colorado}
  \postcode{80305}
}
\email{elfanfa@ucar.edu}

\begin{abstract}
We present the results on the scalability and performance of an open-source, \gls{caf} \gls{mini-app} that solves
several parallel, numerical algorithmns known to dominate the execution of the \gls{icar}~\cite{gutmann2016intermediate}.
The solver employs standard Fortran 2008 features and includes several Fortran 2008 implementations of the collective
subroutines defined in the Committe Draft the upcoming Fortran 2015 standard.  The ability of \gls{caf} to run atop various
communication layers and the increasing compiler support for \gls{caf} facilitated initial evaluations of several compliers,
runtime libraries and hardware platforms.  Results are presented for the GNU, Intel, and Cray compilers, each of which offers
different parallel runtime libraries employing one or more communication layers, including \gls{mpi}, OpenSHMEM, and proprietary
alternatives.  We studied performance on both multi- and many-core processors runnning on disributed-memory systems.  The
results of initial simulations suggest promising scaling behavior across a range of hardware, compiler, and runtime choices
on platforms ranging up to 100,000 cores.
\end{abstract}

%
% The code below was generated by the tool at
% http://dl.acm.org/ccs.cfm
%
\begin{CCSXML}
<ccs2012>
<concept>
<concept_id>10011007.10011006.10011008.10011009.10010175</concept_id>
<concept_desc>Software and its engineering~Parallel programming languages</concept_desc>
<concept_significance>500</concept_significance>
</concept>
<concept>
<concept_id>10010405.10010432.10010437.10010438</concept_id>
<concept_desc>Applied computing~Environmental sciences</concept_desc>
<concept_significance>300</concept_significance>
</concept>
</ccs2012>
\end{CCSXML}

\ccsdesc[500]{Software and its engineering~Parallel programming languages}
\ccsdesc[300]{Applied computing~Environmental sciences}

\keywords{coarray Fortran, computational hydrology, parallel programming}

\maketitle

\section{Introduction}
\subsection{Motivation and Background}
Since the Fortran 2008 standard was published in 2010~\cite{iso2010fortran}, a handful of studies in the open literature
have provided encouraging assessments of Fortran's intrinsic parallel programming model, \gls{caf}, running in research
applications running at scale~\cite{preissl2011multithreaded,garain2015comparing,mozdzynski2015partitioned}.   One of the
most attractive features of \gls{caf} is the ability to express parallel algorithms once within the confines of a single
language statndard without embedding the use of compiler directives or communication layers defined outside the Fortran
Fortran langugage standard (e.g., no \gls{mpi} or OpenMP.  In theory, this should leave the choice of communication layers
as a link-time decision.  Ideally, an application developer writes a standard Fortran-only application, compiles it with any
standard-conforming compiler, and possibly reserves the option to incorporate any one (or more) of several communication layers
at link-time.

Such flexibility poses the computing equivalent of what food author Michael Pollen termed ``The Omnivore's Dilemma:'' if I
belong to a species that owes some of its evolutionary advantage to being able to eat a wide variety of foods, what food is
best for me to eat?  Similiarly, if I can express my parallel algorithm once using \gls{caf} and then consume cycles atop a
multitide of software stacks and hardware platforms, where best to consume cycles and using which supporting software stack?
Here we report the results of an initial study of the current options for compiling, linking, and executing a \gls{mini-app}
designed to be representative of the paralel numerical algorithms and physics employed in the \gls{icar} being developed at
the \gls{ncar}.

\gls{icar} simules the motion of the atmosphere at kilometer length scales and produces flow patterns with a fidelity that is
attractive to the hydrology community, which studies surface water.  The animation in Figure~\ref{figure:icar} depicts flow
patterns at the desired resolution.

\begin{figure*}
  \centering
   % Animate frames at 7 fps created via 'ffmpeg -i qvmovie.mp4 -r 7 frame%d.pngr', add control buttons,
   % and automatically play when the page is viewed:
   \animategraphics[width=0.85\columnwidth,controls,autoplay]{7}{figures/icar/frames/frame}{1}{30}
   \caption{An animation of atmospheric flow patterns simulated by the \href{https://github.com/gutmann/icar}{\gls{icar}}.
      Courtesy of Ethan Gutmann, NCAR.\label{figure:icar}}
\end{figure*}

\subsection{Objectives}

\section{Methodology}
\subsection{Numerical algorithms}
\subsection{Collective subroutines}
\subsection{Compilers, runtimes, and hardware}

\section{Discussion of Results}

\section{Conclusions}


%%%
%%% Sample tables
%%%

%\begin{table}
%  \caption{Frequency of Special Characters}
%  \label{tab:freq}
%  \begin{tabular}{ccl}
%    \toprule
%    Non-English or Math&Frequency&Comments\\
%    \midrule
%    \O & 1 in 1,000& For Swedish names\\
%    $\pi$ & 1 in 5& Common in math\\
%    \$ & 4 in 5 & Used in business\\
%    $\Psi^2_1$ & 1 in 40,000& Unexplained usage\\
%  \bottomrule
%\end{tabular}
%\end{table}

%\begin{table*}
%  \caption{Some Typical Commands}
%  \label{tab:commands}
%  \begin{tabular}{ccl}
%    \toprule
%    Command &A Number & Comments\\
%    \midrule
%    \texttt{{\char'134}author} & 100& Author \\
%    \texttt{{\char'134}table}& 300 & For tables\\
%    \texttt{{\char'134}table*}& 400& For wider tables\\
%    \bottomrule
%  \end{tabular}
%\end{table*}
% end the environment with {table*}, NOTE not {table}!

%It is strongly recommended to use the package booktabs~\cite{Fear05}
%and follow its main principles of typography with respect to tables:
%\begin{enumerate}
%\item Never, ever use vertical rules.
%\item Never use double rules.
%\end{enumerate}
%It is also a good idea not to overuse horizontal rules.


%%%
%%% Sample figures
%%%

%\begin{figure}
%\includegraphics{fly}
%\caption{A sample black and white graphic.}
%\end{figure}

%\begin{figure}
%\includegraphics[height=1in, width=1in]{fly}
%\caption{A sample black and white graphic
%that has been resized with the \texttt{includegraphics} command.}
%\end{figure}

%\begin{figure*}
%\includegraphics{flies}
%\caption{A sample black and white graphic
%that needs to span two columns of text.}
%\end{figure*}
%
%\begin{figure}
%\includegraphics[height=1in, width=1in]{rosette}
%\caption{A sample black and white graphic that has
%been resized with the \texttt{includegraphics} command.}
%\end{figure}

%\end{document}  % This is where a 'short' article might terminate



\appendix
%Appendix A
\section{Code snippets: collective subroutines?}
% This next section command marks the start of
% Appendix B, and does not continue the present hierarchy
\section{Anything else?}

\begin{acks}
  The authors would like to thank the CISL and RAL Visitor Program

  The work is
  supported by the \grantsponsor{GSxxxxx}{National
  Science Foundation China}{http://dx.doi.org/zz.yyyyy/xxxxx} under Grant
  No.:~\grantnum{GSxxxxx}{yyyyyyy}

\end{acks}


\bibliographystyle{ACM-Reference-Format}
\bibliography{bibliography}

\end{document}
