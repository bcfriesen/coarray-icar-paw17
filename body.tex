\section{Introduction}
\subsection{Motivation and Background}
Since the Fortran 2008 standard was published in 2010~\cite{iso2010fortran}, a handful of studies in the open literature
have provided encouraging assessments of Fortran's intrinsic parallel programming model, \gls{caf}, running in research
applications running at scale~\cite{preissl2011multithreaded,garain2015comparing,mozdzynski2015partitioned}.   One of the
most attractive features of \gls{caf} is the ability to express parallel algorithms once within the confines of a single
language statndard without embedding the use of compiler directives or communication layers defined outside the Fortran
Fortran langugage standard (e.g., no \gls{mpi} or OpenMP.  In theory, this should leave the choice of communication layers
as a link-time decision.  Ideally, an application developer writes a standard Fortran-only application, compiles it with any
standard-conforming compiler, and possibly reserves the option to incorporate any one (or more) of several communication layers
at link-time.

Such flexibility poses the computing equivalent of what food author Michael Pollen termed ``The Omnivore's Dilemma:'' if I
belong to a species that owes some of its evolutionary advantage to being able to eat a wide variety of foods, what food is
best for me to eat?  Similiarly, if I can express my parallel algorithm once using \gls{caf} and then consume cycles atop a
multitide of software stacks and hardware platforms, where best to consume cycles and using which supporting software stack?
Here we report the results of an initial study of the current options for compiling, linking, and executing a \gls{mini-app}
designed to be representative of the paralel numerical algorithms and physics employed in the \gls{icar} being developed at
the \gls{ncar}.

\gls{icar} simules the motion of the atmosphere at kilometer length scales and produces flow patterns with a fidelity that is
attractive to the hydrology community, which studies surface water.  The animation in Figure~\ref{figure:icar} depicts flow
patterns at the desired resolution.

\begin{figure*}
  \centering
   % Animate frames at 7 fps created via 'ffmpeg -i qvmovie.mp4 -r 7 frame%d.pngr', add control buttons,
   % and automatically play when the page is viewed:
   \animategraphics[width=0.85\columnwidth,controls,autoplay]{7}{figures/icar/frames/frame}{1}{30}
   \caption{An animation of atmospheric flow patterns simulated by the \href{https://github.com/gutmann/icar}{\gls{icar}}.
      Courtesy of Ethan Gutmann, NCAR.\label{figure:icar}}
\end{figure*}

\subsection{Objectives}

\section{Methodology}
\subsection{Numerical algorithms}
\subsection{Collective subroutines}
\subsection{Compilers, runtimes, and hardware}

We performed the experiments presented here on two supercomputers at at the National Energy Research Scientific Computing Center (NERSC), located at Lawrence Berkeley National Laboratory.
One system is ``Edison,'' a Cray XC30 featuring 5586 nodes with two sockets of 12-core Intel Xeon Processor E5-2695 v2 (``Ivy Bridge''), running at \num{2.4}~\si{\giga\hertz}.
The second system is ``Cori,'' a Cray XC40 which contains \num{12076} compute nodes, spanning two architectures. \num{2388} nodes each have two sockets of 16-core Intel Xeon Processor E5-2698 v3 (``Haswell'') operating at \num{2.3}~\si{\giga\hertz}; the other \num{9688} nodes are single-socket, 68-core Intel Xeon Phi Processor 7250 (``Knights Landing'') at \num{1.4}~\si{\giga\hertz}.
Both Edison and Cori feature the same ``Aries'' interconnect, with a dragonfly topology.
All measurements reported here which ran on Cori were executed on the Xeon Phi nodes, running in the ``quadrant`` NUMA configuration, with the MCDRAM configured as a transparent cache to the DDR4 memory.
We compiled the ICAR code on both systems at NERSC using version 8.6.0 of the Cray Fortran compiler.
In all runs, we ran the code with one image per physical core (68 images per Xeon Phi node on Cori, 24 images per Xeon node on Edison).

\section{Discussion of Results}

\section{Conclusions}


%%%
%%% Sample tables
%%%

%\begin{table}
%  \caption{Frequency of Special Characters}
%  \label{tab:freq}
%  \begin{tabular}{ccl}
%    \toprule
%    Non-English or Math&Frequency&Comments\\
%    \midrule
%    \O & 1 in 1,000& For Swedish names\\
%    $\pi$ & 1 in 5& Common in math\\
%    \$ & 4 in 5 & Used in business\\
%    $\Psi^2_1$ & 1 in 40,000& Unexplained usage\\
%  \bottomrule
%\end{tabular}
%\end{table}

%\begin{table*}
%  \caption{Some Typical Commands}
%  \label{tab:commands}
%  \begin{tabular}{ccl}
%    \toprule
%    Command &A Number & Comments\\
%    \midrule
%    \texttt{{\char'134}author} & 100& Author \\
%    \texttt{{\char'134}table}& 300 & For tables\\
%    \texttt{{\char'134}table*}& 400& For wider tables\\
%    \bottomrule
%  \end{tabular}
%\end{table*}
% end the environment with {table*}, NOTE not {table}!

%It is strongly recommended to use the package booktabs~\cite{Fear05}
%and follow its main principles of typography with respect to tables:
%\begin{enumerate}
%\item Never, ever use vertical rules.
%\item Never use double rules.
%\end{enumerate}
%It is also a good idea not to overuse horizontal rules.


%%%
%%% Sample figures
%%%

%\begin{figure}
%\includegraphics{fly}
%\caption{A sample black and white graphic.}
%\end{figure}

%\begin{figure}
%\includegraphics[height=1in, width=1in]{fly}
%\caption{A sample black and white graphic
%that has been resized with the \texttt{includegraphics} command.}
%\end{figure}

%\begin{figure*}
%\includegraphics{flies}
%\caption{A sample black and white graphic
%that needs to span two columns of text.}
%\end{figure*}
%
%\begin{figure}
%\includegraphics[height=1in, width=1in]{rosette}
%\caption{A sample black and white graphic that has
%been resized with the \texttt{includegraphics} command.}
%\end{figure}

%\end{document}  % This is where a 'short' article might terminate



\appendix
%Appendix A
\section{Code snippets: collective subroutines?}
% This next section command marks the start of
% Appendix B, and does not continue the present hierarchy
\section{Anything else?}

\begin{acks}
  The authors would like to thank the CISL and RAL Visitor Program

  The work is
  supported by the \grantsponsor{GSxxxxx}{National
  Science Foundation China}{http://dx.doi.org/zz.yyyyy/xxxxx} under Grant
  No.:~\grantnum{GSxxxxx}{yyyyyyy}

\end{acks}
