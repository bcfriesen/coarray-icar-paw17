\newglossaryentry{coarray}{name={coarray},description={a variable that may be referenced or defined by multiple images (replicas) of a Fortran program; the portion of a coarray that is local to a given image may be a scalar or an array}}
\newglossaryentry{refactoring}{name={refactoring},description={changing the decomposition of a code without changing its external behavior}}
\newglossaryentry{image}{name={image},description={instance of a Fortran program}}
\newglossaryentry{exascale}{name={exascale},description={Exascale computing will provide exaFLOPS performance, which refers to $10^{18}$ FLoating-point OPerations per Second (FLOPS)}}
\newglossaryentry{pgas}{name={PGAS},description={Partitioned Global Address Space, a parallel programming model in which each unit of thread or process or other basic unit of execution contains some portion of a data structure and knows how to access each other execution unit's portion \shortcite{pacheco2011introduction}}}

\newglossaryentry{designpattern}{name={design pattern},description={A common solution to a recurring problem in software design}}
\newglossaryentry{objectorienteddesign}{name={object-oriented design},description={Schematic descriptions of software structure and behavior stressing the relationships between data structures and between data structures and the procedures for operating on the data structures}}
\newglossaryentry{objectorientedprogramming}{name={object-oriented programming},description={A computer program construction philosophy wherein the programmer couples discrete packages of data with a set of procedures for manipulating those data, emphasizing four basic principles: encapsulation, information-hiding, polymorphism, and inheritance}}
\newglossaryentry{softwaredesign}{name={software design},description={A schematic description of the organization and behavior of a computer program}}
\newglossaryentry{encapsulation}{name={encapsulation},description={Bundling data of various intrinsic or programmer-defined types together with procedures for operating on those data.  Fortran supports encapsulation via the definition of derived-type components and type-bound procedure}}
\newglossaryentry{informationhiding}{name={information hiding},description={Limiting access to data or procedures.  Fortran supports information hiding through the ``private'' attribute, the application of which limits the accessibility of data and procedures to code in the same module where the data or procedures are defined}}
\newglossaryentry{class}{name={class},description={An extensible type that encapsulates data along with procedures for operating on those data.  Fortran supports classes via extensible derived types.  An instance of an extensible derived type can be declared with the keyword ``class'' if it is a subprogram dummy argument or if it has either allocatable or pointer attribute}}
\newglossaryentry{derivedtype}{name={derived type},description={A user-defined Fortran type. Fortran user-defined types are by default extensible}}
\newglossaryentry{aggregation} {name={aggregation},description={A ``has a'' or ``whole/part'' relationship between classes wherein an instance of the class representing the ``whole'' encapsulates (has) one or more instances of the class representing the part(s).  Fortran supports aggregation via the components of derived types}}
\newglossaryentry{composition} {name={composition},description={A special case of aggregation wherein the lifetimes (from construction to destruction) of the whole and part coincide}}
\newglossaryentry{inheritance}{name={inheritance},description={An special case of composition which a child class supports procedure invocations as its parent class either by delegating calls to the parent's implementations of those procedures or by overriding the parents' procedures.  Inheritance often referred to as an ``is a'' relationship or as subclassing.  OOP languages support inheritance by automating the insertion of the parent into the child and by automating the procedural delegations.  Fortran supports inheritance via type extension}}
\newglossaryentry{polymorphism}{name={polymorphism},description={The ability for one named type or one named procedure to reference many different types or procedures, respectively, and to resolve any ambiguities in referencing either at compile-time (in the case of \gls{staticpolymorphism}) or at runtime (in the case of \gls{dynamicpolymorphism})}}
\newglossaryentry{staticpolymorphism}{name={static polymorphism},description={A compile-time technology in which the compiler resolves the invocation of a generic procedure name into one of many actual procedure names based on the type, kind, and rank of the procedure's arguments at the point of invocation}}
\newglossaryentry{dynamicpolymorphism}{name={dynamic polymorphism},description={A run-time technology in which the compiler resolves the invocation of a type-bound procedure name into one of many actual procedure names based on the dynamic type of the object on which the procedure is invoked.  The dynamic type may be either the type named in the declaration of the object or any type that extends the named type}}
\newglossaryentry{typeextension}{name={type extension},description={A mechanism for designating that one derived type extends another and thereby inherits the components and type-bound procedures of the latter type}}
\newglossaryentry{instantiate}{name={instantiate},description={To construct an object as an instance of a class.  This typically involves allocating any resources the object requires (e.g., memory) and initializing the object's data.  A mechanism for designating that one derived type extends another and thereby inherits the components and type-bound procedures of the latter type}}
\newglossaryentry{warp}{name={warp},description={A group of threads in a CUDA-enabled device, treated as a unit by the scheduler}}
\newglossaryentry{dummyargument}{name={dummy argument},description={A ``entity whose identifier appears in a dummy argument list in a function or subroutine$\ldots$ or whose name can be used as an argument keyword in a reference to an intrinsic procedure or a procedure in an intrinsic module''~\shortcite{iso2010information}.}}
\newglossaryentry{literalconstant}{name={literal constant},description={A ``constant that does not have a name.''~\shortcite{iso2010information} }}
\newglossaryentry{actualargument}{name={actual argument},description={An ``entity that appears in a procedure reference.''~\shortcite{iso2010information} In the current book, the entity is an expression or a variable}}
\newglossaryentry{defaultinitialization}{name={default initialization},description={A ``mechanism for automatically initializing pointer components to have a defined pointer association status, and nonpointer components to have a particular value .''~\shortcite{iso2010information}.}}
\newglossaryentry{implicitinterface}{name={implicit interface},description={The ``interface of a procedure that includes only the type and type parameters of a function result''~\shortcite{iso2010information}.}}
\newglossaryentry{explicitinterface}{name={explicit interface},description={The ``interface of a procedure that includes all the characteristics of the procedure and names for its dummy arguments except for asterisk dummy arguments.''~\shortcite{iso2010information}.}}
\newglossaryentry{specificinterfaceblock}{name={specific interface block},description={An ``interface block with no generic-spec or ABSTRACT keyword; collection of interface bodies that specify the interfaces of procedures.'' A generic-spec can be a generic name, an operator, or defined input/output generic-spec~\shortcite{iso2010information}.}}
\newglossaryentry{internalsubprogram}{name={internal subprogram},description={A ``subprogram that is contained in a main program or another subprogram.''~\shortcite{iso2010information}.}}
\newglossaryentry{modulesubprogram}{name={module subprogram},description={A ``subprogram that is contained in a module or submodule but is not an internal subprogram.''~\shortcite{iso2010information}.}}
\newglossaryentry{subprogram}{name={subprogram},description={A function or subroutine.~\shortcite{iso2010information}}}
\newglossaryentry{intrinsic}{name={intrinsic},description={Describing a ``type, procedure, module, assignment, operator, or input/output operation defined in [the Fortran standard] and accessible without further definition or specification, or a procedure or module provided by a [compiler] but not defined in'' the standard.''.~\shortcite{iso2010information}.}}
\newglossaryentry{openmp}{name={OpenMP},description={A parallel programming model comprised of compiler directives, library routines, and environment variables define an application program interface for parallelism in C, C++ and Fortran programs. \shortcite{openmp2015}}.}

\newglossaryentry{mpi}{name={MPI},description={The Message Passing Interface, a language-independent communication protocol for programming parallel computers~\shortcite{gropp1999using}}.}
\newglossaryentry{overloading}{name={operator overloading},description={Adding meaning to an intrinsic operator by providing one or more functions for the compiler to invoke when the operator is applied to user-defined derived types.}}

